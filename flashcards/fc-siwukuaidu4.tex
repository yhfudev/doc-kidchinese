%# -*- coding: utf-8 -*-
%!TEX encoding = UTF-8 Unicode
%!TEX TS-program = xelatex
% vim:ts=4:sw=4
%
% 以上设定默认使用 XeLaTex 编译,并指定 Unicode 编码,供 TeXShop 自动识别
\documentclass[avery5371,grid]{flashcards}


\cardfrontstyle[\large\slshape]{headings}
\cardbackstyle{empty}
%\cardfrontfoot{中文}
\cardfrontfoot{}%四五快读4

\newcommand{\doctitle}{四五快读 4}
\newcommand{\docauthor}{Yunhui Fu}
\newcommand{\dockeywords}{{中文}{四五快读}}
\newcommand{\docsubject}{}

% debug
\newcommand{\SecDanzi}[1]{#1} % 单字
\newcommand{\SecCi}[1]{#1} % 词
\newcommand{\SecJuzi}[1]{#1} % 句子

%\renewcommand{\SecDanzi}[1]{} % 单字
%\renewcommand{\SecCi}[1]{} % 词
%\newcommand{\comments}[1]{} % 句子


%# -*- coding: utf-8 -*-
%!TEX encoding = UTF-8 Unicode
%!TEX TS-program = xelatex
% vim:ts=4:sw=4
%
% 以上设定默认使用 XeLaTex 编译,并指定 Unicode 编码,供 TeXShop 自动识别


\newcommand\mymainfont{Noto Serif}%{Times New Roman} %{DejaVu Serif}
\newcommand\mymonofont{DejaVu Sans Mono}%{FreeMono} %{WenQuanYi Micro Hei Mono} %{Monaco}
\newcommand\myboldfont{DejaVu Sans Mono} %{WenQuanYi Micro Hei Mono}%{AR PL UKai CN}%{YaHei Consolas Hybrid}%{黑体}%{標楷體}
\newcommand\mysansfont{DejaVu Sans}%{FreeSans}
\newcommand\myitalicfont{DejaVu Serif}%{Times New Roman} %{Garamond}

\newcommand\mycjkboldfont{Microsoft YaHei} %{WenQuanYi Micro Hei Mono}%{Adobe Heiti Std}%{AR PL UKai CN}%{YaHei Consolas Hybrid}%{黑体}%{標楷體}
\newcommand\mycjkitalicfont{全字庫正楷體} %{Adobe Kaiti Std}
\newcommand\mycjkmainfont{WenyueType GutiFangsong (Noncommercial Use)}%{全字庫正楷體} %{Adobe Song Std} %{AR PL UMing CN}%{仿宋}%{宋体}%{新宋体}%{文鼎PL新宋}%
\newcommand\mycjksansfont{MingLiU} %{Adobe Ming Std}
\newcommand\mycjkmonofont{DejaVu Sans YuanTi Mono}%{WenQuanYi Micro Hei Mono}%{AR PL UMing CN}%{WenQuanYi Micro Hei Mono}


%\usepackage[nofonts]{ctex} %adobefonts
\usepackage[adobefonts]{ctex} %adobefonts
%\usepackage[fallback]{xeCJK}


\newCJKfontfamily{\mykaiti}{全字庫正楷體}%{AR PL UKai TW}%{全字庫正楷體}
\newCJKfontfamily{\myfangsong}{WenyueType GutiFangsong (Noncommercial Use)}



\usepackage{ifthen}
\usepackage{ifpdf}
\usepackage{ifxetex}
\usepackage{ifluatex}


\usepackage{color}
\usepackage[rgb]{xcolor}



\xeCJKsetup{AutoFallBack = true} % you have to use this, since fallback won't work as the xeCJK option after the ctex

\PassOptionsToPackage{
    BoldFont,  % 允許粗體
    SlantFont, % 允許斜體
    CJKnumber,
    CJKtextspaces,
    }{xeCJK}
\defaultfontfeatures{Mapping=tex-text} % 如果沒有它,會有一些 tex 特殊字符無法正常使用,比如連字符。

\xeCJKsetup {
    CheckSingle = true,
    AutoFakeBold = false,
AutoFakeSlant = false,
    CJKecglue = {},
    PunctStyle = kaiming,
KaiMingPunct+ = {:;},
}

\PassOptionsToPackage{CJKchecksingle}{xeCJK}
%\defaultCJKfontfeatures{Scale=0.5}
%\LoadClass[c5size,openany,nofonts]{ctexbook}
\XeTeXlinebreaklocale "zh"                      % 重要,使得中文可以正確斷行!
\XeTeXlinebreakskip = 0pt plus 1pt minus 0.1pt  % 给予TeX断行一定自由度
\linespread{1.3}                                % 1.3倍行距

\setCJKmainfont[BoldFont=\mycjkboldfont, ItalicFont=\mycjkitalicfont]{\mycjkmainfont}
\setCJKsansfont{\mycjksansfont}%{Adobe Ming Std} %{AR PL UMing CN} %{Microsoft YaHei}
\setCJKmonofont{\mycjkmonofont}



\ifxetex % xelatex
\else
    %The cmap package is intended to make the PDF files generated by pdflatex "searchable and copyable" in acrobat reader and other compliant PDF viewers.
    \usepackage{cmap}%
\fi
% ============================================
% Check for PDFLaTeX/LaTeX
% ============================================
\newcommand{\outengine}{xetex}
\newif\ifpdf
\ifx\pdfoutput\undefined
  \pdffalse % we are not running PDFLaTeX
  \ifxetex
    \renewcommand{\outengine}{xetex}
  \else
    \renewcommand{\outengine}{dvipdfmx}
  \fi
\else
  \pdfoutput=1 % we are running PDFLaTeX
  \pdftrue
  \usepackage{thumbpdf}
  \renewcommand{\outengine}{pdftex}
  \pdfcompresslevel=9
\fi
\usepackage[\outengine,
    bookmarksnumbered, %dvipdfmx
    %% unicode, %% 不能有unicode选项,否则bookmark会是乱码
    colorlinks=true,
    citecolor=red,
    urlcolor=blue,        % \href{...}{...} external (URL)
    filecolor=red,      % \href{...} local file
    linkcolor=black, % \ref{...} and \pageref{...}
    breaklinks,
    pdftitle={\doctitle},
    pdfauthor={\docauthor},
    pdfsubject={\docsubject},
    pdfkeywords={\dockeywords},
    pdfproducer={Latex with hyperref},
    pdfcreator={pdflatex},
    %%pdfadjustspacing=1,
    pdfborder=1,
    pdfpagemode=UseNone,
    pagebackref,
    bookmarksopen=true]{hyperref}

% --------------------------------------------
% Load graphicx package with pdf if needed 
% --------------------------------------------
\ifxetex    % xelatex
    \usepackage{graphicx}
\else
    \ifpdf
        \usepackage[pdftex]{graphicx}
        \pdfcompresslevel=9
    \else
        \usepackage{graphicx} % \usepackage[dvipdfm]{graphicx}
    \fi
\fi

%\newCJKfontfamily{\yanti}{\mycjkmainfont}
%\newenvironment{shicibody}{%
%\begin{verse}\centering\yanti\large\hspace{13pt}}{\end{verse}}


\usepackage{xstring}

%The baseline-skip should be set to roughly 1.2x the font size.
%\fontsize{size}{baselineskip}
%\fontsize{50}{60}

% 计算所能容纳的字数
\newcommand\shizishow[1]{
\StrLen{#1}[\mystringlen]
%\def\mythresh{1}
    \ifthenelse{\mystringlen < 2}{
        {\fontsize{130}{156} #1}
    }{ \ifthenelse{\mystringlen < 3}{
        {\fontsize{110}{122} #1}%{\fontsize{100}{120} #1}
    }{ \ifthenelse{\mystringlen < 4}{
        {\fontsize{78}{94} #1}%{\fontsize{90}{108} #1}
    }{ \ifthenelse{\mystringlen < 5}{
        {\fontsize{55}{66} #1}
    }{ \ifthenelse{\mystringlen < 7}{
        {\fontsize{70}{84} #1}
    }{ \ifthenelse{\mystringlen < 9}{
        {\fontsize{55}{66} #1}
    }{ \ifthenelse{\mystringlen < 16}{
        {\zihao{0} #1} %5x4=20
    }{ \ifthenelse{\mystringlen < 49}{
        {\zihao{1} #1} % 8x6=48
    }{ \ifthenelse{\mystringlen < 55}{
        {\Huge #1} % 9x6=54
    }{ \ifthenelse{\mystringlen < 61}{
        {\zihao{2} #1} % 10x6=60
    }{ \ifthenelse{\mystringlen < 78}{
        {\huge #1} % 11x7=77
    }{ \ifthenelse{\mystringlen < 105}{
        {\LARGE #1} % 13x8=104
    }{ \ifthenelse{\mystringlen < 113}{
        {\zihao{3} #1} % 14x8=112
    }{ \ifthenelse{\mystringlen < 129}{
        {\Large #1} % 16x8=128 %{\zihao{4} #1} % 16x8=128
    }{ \ifthenelse{\mystringlen < 153}{
        {\large #1} % 19x8=152, or 19x9=171
    }{ \ifthenelse{\mystringlen < 198}{
        {\zihao{5} #1} % 22x9=198
    }{ \ifthenelse{\mystringlen < 207}{
        {\normalsize #1} % 23x9=207
    }{ \ifthenelse{\mystringlen < 234}{
        {\small #1} % 26x9=234
    }{ \ifthenelse{\mystringlen < 261}{
        {\footnotesize #1} % 29x9=261
    }{ \ifthenelse{\mystringlen < 279}{
        {\zihao{6} #1} % 31x9=279
    }{ \ifthenelse{\mystringlen < 297}{
        {\scriptsize #1} % 33x9=297
    }{ \ifthenelse{\mystringlen < 378}{
        {\zihao{7} #1} % 42x9=378
    }{
        {\tiny #1} % 46x9=414 %{\zihao{8} #1} % 46x9=414
    }}}}}}}}}}}}}}}}}}}}}}
}

% 计算包含拼音时所能容纳的字数
\newcommand\shizipy[1]{
\StrLen{#1}[\mystringlen]
\xpinyin*{
    \ifthenelse{\mystringlen < 6}{
        {\zihao{0} #1} %5x4=20
    }{ \ifthenelse{\mystringlen < 18}{
        {\zihao{1} #1} % 8x6=48
    }{ \ifthenelse{\mystringlen < 22}{
        {\Huge #1} % 9x6=54
    }{ \ifthenelse{\mystringlen < 31}{
        {\zihao{2} #1} % 10x6=60
    }{ \ifthenelse{\mystringlen < 34}{
        {\huge #1} % 11x7=77
    }{ \ifthenelse{\mystringlen < 53}{
        {\LARGE #1} % 13x8=104
    }{ \ifthenelse{\mystringlen < 57}{
        {\zihao{3} #1} % 14x8=112
    }{ \ifthenelse{\mystringlen < 65}{
        {\Large #1} % 16x8=128 %{\zihao{4} #1} % 16x8=128
    }{ \ifthenelse{\mystringlen < 77}{
        {\large #1} % 19x8=152, or 19x9=171
    }{ \ifthenelse{\mystringlen < 89}{
        {\zihao{5} #1} % 22x9=198
    }{ \ifthenelse{\mystringlen < 93}{
        {\normalsize #1} % 23x9=207
    }{ \ifthenelse{\mystringlen < 105}{
        {\small #1} % 26x9=234
    }{ \ifthenelse{\mystringlen < 117}{
        {\footnotesize #1} % 29x9=261
    }{ \ifthenelse{\mystringlen < 125}{
        {\zihao{6} #1} % 31x9=279
    }{ \ifthenelse{\mystringlen < 133}{
        {\scriptsize #1} % 33x9=297
    }{ \ifthenelse{\mystringlen < 169}{
        {\zihao{7} #1} % 42x9=378
    }{
        {\tiny #1} % 46x9=414 %{\zihao{8} #1} % 46x9=414
    }}}}}}}}}}}}}}}}
}
}

% 计算在comment中(三字经)能容纳的字数
\newcommand\sanzicomments[1]{
\StrLen{#1}[\mystringlen]
%\def\mythresh{1}
    \ifthenelse{\mystringlen < 7}{
        {\zihao{0} #1} %5x4=20
    }{ \ifthenelse{\mystringlen < 22}{
        {\zihao{1} #1} % 8x6=48
    }{ \ifthenelse{\mystringlen < 25}{
        {\Huge #1} % 9x6=54
    }{ \ifthenelse{\mystringlen < 38}{
        {\zihao{2} #1} % 10x6=60
    }{ \ifthenelse{\mystringlen < 42}{ %
        {\huge #1} % 11x7=77
    }{ \ifthenelse{\mystringlen < 63}{
        {\LARGE #1} % 13x8=104
    }{ \ifthenelse{\mystringlen < 64}{
        {\zihao{3} #1} % 14x8=112
    }{ \ifthenelse{\mystringlen < 87}{
        {\Large #1} % 16x8=128 %{\zihao{4} #1} % 16x8=128
    }{ \ifthenelse{\mystringlen < 125}{
        {\large #1} % 19x8=152, or 19x9=171
    }{ \ifthenelse{\mystringlen < 173}{
        {\zihao{5} #1} % 22x9=198
    }{ \ifthenelse{\mystringlen < 205}{
        {\normalsize #1} % 23x9=207
    }{ \ifthenelse{\mystringlen < 232}{
        {\small #1} % 26x9=234
    }{ \ifthenelse{\mystringlen < 259}{
        {\footnotesize #1} % 29x9=261
    }{ \ifthenelse{\mystringlen < 277}{
        {\zihao{6} #1} % 31x9=279
    }{ \ifthenelse{\mystringlen < 295}{
        {\scriptsize #1} % 33x9=297
    }{ \ifthenelse{\mystringlen < 376}{
        {\zihao{7} #1} % 42x9=378
    }{
        {\tiny #1} % 46x9=414 %{\zihao{8} #1} % 46x9=414
    }}}}}}}}}}}}}}}}
}



% 诗词
\newcommand\shici[5]{
\begin{flashcard}[#1 -- #2 (#3) #4]{%
{\mykaiti \Large {#5}} %\\
}
\vspace*{\stretch{1}}
\begin{center}

{\mykaiti \zihao{1} {#2}}\\ \vspace*{\stretch{.5}}
{\large(#3)}\\ \vspace*{\stretch{.25}}
{\LARGE #4}

\end{center}
\vspace*{\stretch{1}}
\end{flashcard}
}



% 识字

%\usepackage[overlap,CJK]{ruby}
\usepackage{xpinyin}

% 如果字数多,则略写
\newcommand\shizititle[2]{
  \StrLen{(#1) #2}[\mystringlen]%
  \ifthenelse{\mystringlen > 24}{
    \StrLeft{(#1) #2}{20}......\StrRight{(#1) #2}{4}
  }{
  (#1) #2
  }
}
\newcommand\shizi[3]{
\begin{flashcard}[\shizititle{#1}{#2}]{ %
{\mykaiti \shizipy{#2}}

{\myfangsong \normalsize #3} %
} %
\vspace*{\stretch{1}}
\begin{center}
%ēéěè
%\ruby{莉}{li}
%\xpinyin*{床前明月光,疑是地上霜。举头望明月,低头思故乡。}
%\begin{pinyinscope}
%床前明月光,疑是地上霜。举头望明月,低头思故乡。
%\end{pinyinscope}

{\mykaiti \shizishow{#2}}

\end{center}
\vspace*{\stretch{1}}
\end{flashcard}
}




% 三字经
\newcommand\sanzijing[3]{
\begin{flashcard}[]{\mykaiti \fontsize{26}{\baselineskip}\selectfont \xpinyin*{#1}}%{\Huge \xpinyin*{#1}}
\vspace*{\stretch{1}}
%\begin{center}
%\normalsize
{\myfangsong
\sanzicomments{
【解释】 #2

〖解读〗 #3
}}
%\end{center}
\vspace*{\stretch{1}}
\end{flashcard}
}





\newcommand\docshowcopyright{
\begin{flashcard}[Copyright \& License]{Copyright \copyright \, 2014 Yunhui Fu\\
Some rights reserved.}
\vspace*{\stretch{1}}
These flashcards and the accompanying \LaTeX \, source code are licensed
under a Creative Commons Attribution--NonCommercial--ShareAlike 2.5 License.  
For more information, see creativecommons.org.  You can contact the author at:
\begin{center}
\begin{small}\tt yhfudev at gmail com\end{small}

\medskip
File last updated on \today, \\
at \currenttime
\end{center}
\vspace*{\stretch{1}}
\end{flashcard}

\begin{flashcard}[版权申明]{版权所有 \copyright \, 2014 Yunhui Fu\\
有些版权保留.}
\vspace*{\stretch{1}}
闪卡及其 \LaTeX \, 源代码在
署名-相同方式共享 2.5 下保护.
参见 creativecommons.org.  你可以联系作者:
\begin{center}
\begin{small}\tt yhfudev at gmail com\end{small}

\medskip
文件最近更新: \today, \\
\currenttime
\end{center}
\vspace*{\stretch{1}}
\end{flashcard}
}






\newcommand\docshowtitle[3]{
\begin{flashcard}[]{ %
{\mykaiti \Huge{#1}}

{\myfangsong \normalsize #2} %
} %
\vspace*{\stretch{1}}
\begin{center}

{\mykaiti \shizishow{#3}}

\end{center}
\vspace*{\stretch{1}}
\end{flashcard}
}








\usepackage{datetime}



\begin{document}


\docshowcopyright
\docshowtitle{\doctitle}{\docauthor}{%
使用双面打印,然后按线剪下。
}


\SecDanzi{
\shizi{4-31}{片}{}
\shizi{4-31}{吹}{}
\shizi{4-31}{浇}{}
\shizi{4-31}{燕}{}
\shizi{4-31}{睡}{}
\shizi{4-31}{醒}{}
\shizi{4-31}{蛙}{}
\shizi{4-31}{呱}{}
\shizi{4-31}{南}{}
\shizi{4-31}{椅}{}
\shizi{4-31}{坐}{}
\shizi{4-31}{身}{}
\shizi{4-31}{吧}{}
\shizi{4-31}{桌}{}
\shizi{4-31}{布}{}
\shizi{4-31}{抱}{}
}

% 宝宝读词语
\SecCi{
\shizi{4-31}{一片}{}
\shizi{4-31}{吹牛}{}
\shizi{4-31}{浇地}{}
\shizi{4-31}{睡梦}{}
\shizi{4-31}{醒了}{}
\shizi{4-31}{牛蛙}{}
\shizi{4-31}{吹风}{}
\shizi{4-31}{浇水}{}
\shizi{4-31}{燕子}{}
\shizi{4-31}{早睡}{}
\shizi{4-31}{醒来}{}
\shizi{4-31}{呱呱叫}{}
\shizi{4-31}{吹气}{}
\shizi{4-31}{浇花}{}
\shizi{4-31}{睡醒}{}
\shizi{4-31}{早起}{}
\shizi{4-31}{青蛙}{}
}

% 宝宝读短文
\SecJuzi{
\shizi{4-31}{“春风吹,春雨下。吹绿了柳树,浇红了桃花。吹来了燕子,浇醒了青蛙。”小朋友们快乐地唱着儿歌。
春天到了,在南方过冬的燕子飞回了它的老家。小燕子们飞来飞去,高兴地问好。睡了一个冬天的青蛙也醒了过来,“呱,呱”地说着话。
春天里,大家都很快乐!
早上,飞来一只小黄鸟,落在门前的树上。它快乐地唱着歌:“春天到了!春天到了!桃花红了,柳树绿了。农民伯伯要下种种地了。”
春天是一年中最好的季节。在春天里,桃树上开出了红红的花,柳树上长出了绿绿的叶,青青的小草从地里长出来,农民伯伯种的种子也长出了小苗。大地一片青绿。天气不冷了,妈妈说要去春游了。
春天真好,我爱春天!}{}
\shizi{4-31}{早上,飞来一只小黄鸟,落在门前的树上。它快乐地唱着歌:“春天到了!春天到了!桃花红了,柳树绿了。农民伯伯要下种种地了。”
春天是一年中最好的季节。在春天里,桃树上开出了红红的花,柳树上长出了绿绿的叶,青青的小草从地里长出来,农民伯伯种的种子也长出了小苗。大地一片青绿。天气不冷了,妈妈说要去春游了。
春天真好,我爱春天!}{}
}


\SecDanzi{
\shizi{4-32}{摔}{}
\shizi{4-32}{声}{}
\shizi{4-32}{谁}{}
\shizi{4-32}{呢}{}
\shizi{4-32}{认}{}
\shizi{4-32}{原}{}
\shizi{4-32}{痛}{}
\shizi{4-32}{喊}{}
}

%宝宝读词语
\SecCi{
\shizi{4-32}{南边}{}
\shizi{4-32}{椅子}{}
\shizi{4-32}{身子}{}
\shizi{4-32}{来吧}{}
\shizi{4-32}{桌子}{}
\shizi{4-32}{花布}{}
\shizi{4-32}{南方}{}
\shizi{4-32}{请坐}{}
\shizi{4-32}{身上}{}
\shizi{4-32}{去吧}{}
\shizi{4-32}{书桌}{}
\shizi{4-32}{桌布}{}
\shizi{4-32}{南面}{}
\shizi{4-32}{坐下}{}
\shizi{4-32}{身高}{}
\shizi{4-32}{好吧}{}
\shizi{4-32}{方桌}{}
\shizi{4-32}{抱着}{}
\shizi{4-32}{南风}{}
\shizi{4-32}{坐着}{}
\shizi{4-32}{身边}{}
\shizi{4-32}{吃吧}{}
\shizi{4-32}{桌椅}{}
\shizi{4-32}{抱起}{}
}

% 宝宝读短文
\SecJuzi{
\shizi{4-32}{妈妈不在家,燕燕哭着要找妈妈。
小椅子跑过来说:“燕燕不哭,燕燕不哭。坐在我身上,我来做你的妈妈。”燕燕看着小椅子,还是哭。
小桌子跑过来说:“燕燕不哭,燕燕不哭。坐在我身上,我来做你的妈妈。”燕燕看着小桌子,还是哭。
布猴子跑过来说:“燕燕不哭,燕燕不哭。把我跑起来,你来做我的妈妈。”燕燕看着不猴子,不哭了。
她抱起布猴子,和它玩起了“过家家”的游戏。}{}
\shizi{4-32}{我家有我的小桌子和小椅子。我的书和文具盒都放在桌子上面。我有很多好看的书,文具盒还有笔、小刀和尺子。我天天都坐在桌子后面的小椅子上看书,画画。
我爱我的小桌子和小椅子。它们都是我的好朋友。}{}
}


\SecDanzi{
\shizi{4-33}{狼}{}
\shizi{4-33}{啦}{}
\shizi{4-33}{赶}{}
\shizi{4-33}{救}{}
\shizi{4-33}{假}{}
\shizi{4-33}{掉}{}
\shizi{4-33}{路}{}
\shizi{4-33}{碰}{}
}

%宝宝读词语
\SecCi{
\shizi{4-33}{摔东西}{}
\shizi{4-33}{大声}{}
\shizi{4-33}{笑声}{}
\shizi{4-33}{哭声}{}
\shizi{4-33}{谁的}{}
\shizi{4-33}{认得}{}
\shizi{4-33}{摔了}{}
\shizi{4-33}{小声}{}
\shizi{4-33}{风声}{}
\shizi{4-33}{读书声}{}
\shizi{4-33}{谁呀}{}
\shizi{4-33}{认生}{}
\shizi{4-33}{摔打}{}
\shizi{4-33}{歌声}{}
\shizi{4-33}{雨声}{}
\shizi{4-33}{是谁}{}
\shizi{4-33}{认字}{}
\shizi{4-33}{认真}{}
\shizi{4-33}{草原}{}
\shizi{4-33}{好痛}{}
\shizi{4-33}{痛快}{}
\shizi{4-33}{大喊}{}
\shizi{4-33}{喊声}{}
\shizi{4-33}{高原}{}
\shizi{4-33}{心痛}{}
\shizi{4-33}{头痛}{}
\shizi{4-33}{喊人}{}
\shizi{4-33}{原来}{}
\shizi{4-33}{痛哭}{}
\shizi{4-33}{喊叫}{}
\shizi{4-33}{喊着}{}
}

% 宝宝读故事
\SecJuzi{
\shizi{4-33}{%小明生气了
小明生气了,就把小人书摔到地上,又把他的玩具——小猴子、小狗、小猫、兔子、白马都摔到了地上。
过了一会儿,小明听到了哭声,哭声不大。
“是睡在哭呢?”
小明打开门,想找找是谁在哭。
没有看见哭的人。
小明回到家里,又听见了哭声。认真听听,哭声是从地上来的。
“是谁在地上哭呢?”
原来是地上的小人书、小猴子、小狗、小猫、兔子和白马。
它们一边哭一边说:“小明把我们摔到地上,摔得好痛!”
小朋友,你生气时,是不是也摔东西,也摔书,也摔玩具?告诉你,把谁摔到地上,都是很痛的。
}{}
}


\SecDanzi{
\shizi{4-34}{哪}{}
\shizi{4-34}{呀}{}
\shizi{4-34}{两}{}
\shizi{4-34}{逃}{}
\shizi{4-34}{走}{}
\shizi{4-34}{她}{}
\shizi{4-34}{点}{}
\shizi{4-34}{音}{}
}

%宝宝读词语
\SecCi{
\shizi{4-34}{来啦}{}
\shizi{4-34}{吃啦}{}
\shizi{4-34}{赶快}{}
\shizi{4-34}{赶不上}{}
\shizi{4-34}{赶着}{}
\shizi{4-34}{救星}{}
\shizi{4-34}{假话}{}
\shizi{4-34}{狼狗}{}
\shizi{4-34}{去啦}{}
\shizi{4-34}{好啦}{}
\shizi{4-34}{赶车}{}
\shizi{4-34}{赶走}{}
\shizi{4-34}{救火}{}
\shizi{4-34}{真假}{}
\shizi{4-34}{大灰狼}{}
\shizi{4-34}{走啦}{}
\shizi{4-34}{快啦}{}
\shizi{4-34}{赶路}{}
\shizi{4-34}{赶得上}{}
\shizi{4-34}{救人}{}
\shizi{4-34}{假的}{}
\shizi{4-34}{掉了}{}
\shizi{4-34}{吃掉}{}
\shizi{4-34}{掉头}{}
\shizi{4-34}{马路}{}
\shizi{4-34}{路人}{}
\shizi{4-34}{碰见}{}
\shizi{4-34}{碰头}{}
\shizi{4-34}{假山}{}
\shizi{4-34}{掉到}{}
\shizi{4-34}{跑掉}{}
\shizi{4-34}{大路}{}
\shizi{4-34}{路上}{}
\shizi{4-34}{路面}{}
\shizi{4-34}{碰到}{}
\shizi{4-34}{掉东西}{}
\shizi{4-34}{走掉}{}
\shizi{4-34}{掉车}{}
\shizi{4-34}{公路}{}
\shizi{4-34}{路口}{}
\shizi{4-34}{碰上}{}
}

% 宝宝读故事
\SecJuzi{
\shizi{4-34}{%狼来了
从前,有一个小孩子,天天在山上放羊。
一天,他在山上大声喊:“狼来啦!狼来啦!”
山下的人听见了,赶快跑上山来救他。放羊的孩子笑了,他说:“没有狼,没有狼。我是说着玩的。”
过了一天,他有在山上喊:“狼来啦!狼来啦!”
山下的人听见了,又赶快跑上山来救他。放羊的孩子又笑了,他说:“没有狼,没有狼。我是说着玩的。”
有一天,狼真的来了。放羊的孩子在山上大声喊:“狼来啦!狼来啦!狼真的来啦!”
山下的人听见了,说:“小孩子又在说假话了。”
人们没有来打狼。狼就吃掉了小孩子的羊。}{}
}


\SecDanzi{
\shizi{4-35}{可}{}
\shizi{4-35}{伸}{}
\shizi{4-35}{缝}{}
\shizi{4-35}{夹}{}
\shizi{4-35}{根}{}
\shizi{4-35}{棍}{}
\shizi{4-35}{丢}{}
\shizi{4-35}{灰}{}
}


%宝宝读词语
\SecCi{
\shizi{4-35}{哪里}{}
\shizi{4-35}{来呀}{}
\shizi{4-35}{好呀}{}
\shizi{4-35}{两边}{}
\shizi{4-35}{逃跑}{}
\shizi{4-35}{走路}{}
\shizi{4-35}{哪个}{}
\shizi{4-35}{去呀}{}
\shizi{4-35}{走呀}{}
\shizi{4-35}{四两}{}
\shizi{4-35}{逃学}{}
\shizi{4-35}{走好}{}
\shizi{4-35}{哪儿}{}
\shizi{4-35}{吃呀}{}
\shizi{4-35}{两个}{}
\shizi{4-35}{逃走}{}
\shizi{4-35}{逃掉}{}
\shizi{4-35}{走到}{}
\shizi{4-35}{她的}{}
\shizi{4-35}{一点}{}
\shizi{4-35}{点名}{}
\shizi{4-35}{点子}{}
\shizi{4-35}{声音}{}
\shizi{4-35}{音乐}{}
\shizi{4-35}{给她}{}
\shizi{4-35}{十点}{}
\shizi{4-35}{点头}{}
\shizi{4-35}{快点}{}
\shizi{4-35}{高音}{}
\shizi{4-35}{走来走去}{}
\shizi{4-35}{快走}{}
\shizi{4-35}{她们}{}
\shizi{4-35}{点心}{}
\shizi{4-35}{点火}{}
\shizi{4-35}{高点}{}
\shizi{4-35}{口音}{}
}

% 宝宝读故事
\SecJuzi{
\shizi{4-35}{%聪明勇敢的公鸡
公鸡在路上碰见了老狼。老狼心里很高兴,它想:“好呀,有鸡吃了。我要吃掉它。”
“好开心呀,有鸡吃啦!好开心呀,有鸡吃啦!”老狼高兴得真想唱歌。
老狼问:“公鸡,你上哪儿去呀?”
公鸡说:“我看我的朋友去。”
老狼说:“我们两个做朋友,在一起走。好吗?”
公鸡说:“哪里只有我们两个呀!你看,后面不是还有我的一个好朋友吗!”
老狼问:“谁呀?”
公鸡说:“大黄狗呀!又高又大的大黄狗。”
老狼一听有狗在后面,头都没有回,就飞快地逃走了。
聪明又勇敢的公鸡哈哈大笑起来。
小朋友,你说后面真的有大黄狗吗?}{}
\shizi{4-35}{%谁爱吃什么
一天,小猴子请小山羊吃饭,拿出来一大推桃子。
“请吃吧,很好吃呢!”小猴子说。
小山羊不爱吃桃子,就很有礼貌地谢了小猴子,回家去了。
过了一天,小山羊也请小猴子吃饭,它拿出来一大堆青草:“请吃吧,很好吃呢!”
小猴子也不爱吃青草,也很有礼貌地谢了小山羊,回家去了。
山羊爷爷后来告诉小山羊和小猴子:“你们想要高高兴兴地在一起吃饭,就只要把你们爱吃的东西拿来,放在一起。小猴子吃小猴子爱吃的桃子,小山羊吃小山羊爱吃的青草。吃好饭,再在一起玩,不就很开心了吗!”}{}
}


\SecDanzi{
\shizi{4-36}{萝}{}
\shizi{4-36}{卜}{}
\shizi{4-36}{熟}{}
\shizi{4-36}{拔}{}
\shizi{4-36}{拉}{}
\shizi{4-36}{鼠}{}
\shizi{4-36}{咕}{}
\shizi{4-36}{咚}{}
}


%宝宝读词语
\SecCi{
\shizi{4-36}{可是}{}
\shizi{4-36}{可用}{}
\shizi{4-36}{伸出}{}
\shizi{4-36}{伸手}{}
\shizi{4-36}{门缝}{}
\shizi{4-36}{小缝}{}
\shizi{4-36}{夹缝}{}
\shizi{4-36}{缝子}{}
\shizi{4-36}{夹子}{}
\shizi{4-36}{夹住}{}
\shizi{4-36}{夹起来}{}
\shizi{4-36}{一根}{}
\shizi{4-36}{草根}{}
\shizi{4-36}{树根}{}
\shizi{4-36}{根苗}{}
\shizi{4-36}{根子}{}
\shizi{4-36}{棍子}{}
\shizi{4-36}{木棍}{}
\shizi{4-36}{丢了}{}
\shizi{4-36}{丢掉}{}
\shizi{4-36}{丢人}{}
\shizi{4-36}{大灰狼}{}
\shizi{4-36}{灰白}{}
\shizi{4-36}{灰心}{}
}

\SecJuzi{
\shizi{4-36}{%小兔子和大灰狼的故事
}{}
}


\SecDanzi{
\shizi{4-37}{倒}{}
\shizi{4-37}{抬}{}
\shizi{4-37}{晚}{}
\shizi{4-37}{左}{}
\shizi{4-37}{右}{}
\shizi{4-37}{怎}{}
\shizi{4-37}{么}{}
\shizi{4-37}{办}{}
}

%宝宝读词语
\SecCi{
\shizi{4-37}{萝卜}{}
\shizi{4-37}{熟了}{}
\shizi{4-37}{成熟}{}
\shizi{4-37}{背熟了}{}
\shizi{4-37}{吧出来}{}
\shizi{4-37}{拔草}{}
\shizi{4-37}{拔起}{}
\shizi{4-37}{拔河}{}
\shizi{4-37}{拉手}{}
\shizi{4-37}{拉着}{}
\shizi{4-37}{拉住}{}
\shizi{4-37}{拉走}{}
\shizi{4-37}{拉来}{}
\shizi{4-37}{老鼠}{}
\shizi{4-37}{田鼠}{}
\shizi{4-37}{咕咚}{}
}


\SecJuzi{
\shizi{4-37}{%拔萝卜
小叶子种了一个萝卜。萝卜长呀长呀,长得很大很大。萝卜长熟了,可以拔出来吃了。
小叶子就去拔萝卜,拔呀拔呀,拔不出来。
小叶子去把爷爷请来和她一起拔。小叶子拉着萝卜叶,爷爷拉着小叶子,拔呀拔呀,拔不出来。
小叶子又去把奶奶请来和她们一起拔。小叶子拉着萝卜叶,爷爷拉着小叶子,奶奶拉着爷爷,拔呀拔呀,拔不出来。
小叶子又去把爸爸请来和她们一起拔。小叶子拉着萝卜叶,爷爷拉着小叶子,奶奶拉着爷爷,爸爸拉着奶奶,拔呀拔呀,还是拔不出来。
小叶子又去把妈妈请来和她们一起拔。小叶子拉着萝卜叶,爷爷拉着小叶子,奶奶拉着爷爷,爸爸拉着奶奶,妈妈拉着爸爸,拔呀拔呀,还是拔不出来。
小叶子又去把小狗请来和她们一起拔。小叶子拉着萝卜叶,爷爷拉着小叶子,奶奶拉着爷爷,爸爸拉着奶奶,妈妈拉着爸爸,小狗拉着妈妈,拔呀拔呀,还是拔不出来。
小叶子又去把小猫请来和她们一起拔。小叶子拉着萝卜叶,爷爷拉着小叶子,奶奶拉着爷爷,爸爸拉着奶奶,妈妈拉着爸爸,小狗拉着妈妈,小猫拉着小狗,拔呀拔呀,还是拔不出来。
小叶子又去把小老鼠请来和她们一起拔。小叶子拉着萝卜叶,爷爷拉着小叶子,奶奶拉着爷爷,爸爸拉着奶奶,妈妈拉着爸爸,小狗拉着妈妈,小猫拉着小狗,小老鼠拉着小猫。
“一、二、三!”大家用力一起拔。“咕咚”一声,大家都摔倒了,大萝卜也拔出来了。
大家高高兴兴地抬着大萝卜回家了。}{}
}


\SecDanzi{
\shizi{4-38}{知}{}
\shizi{4-38}{道}{}
\shizi{4-38}{午}{}
\shizi{4-38}{这}{}
\shizi{4-38}{座}{}
\shizi{4-38}{洞}{}
\shizi{4-38}{什}{}
\shizi{4-38}{害}{}
}

%宝宝读词语
\SecCi{
\shizi{4-38}{倒了}{}
\shizi{4-38}{倒下}{}
\shizi{4-38}{摔倒}{}
\shizi{4-38}{倒车}{}
\shizi{4-38}{抬着}{}
\shizi{4-38}{抬头}{}
\shizi{4-38}{抬起}{}
\shizi{4-38}{抬走}{}
\shizi{4-38}{晚上}{}
\shizi{4-38}{明晚}{}
\shizi{4-38}{晚了}{}
\shizi{4-38}{左手}{}
\shizi{4-38}{左边}{}
\shizi{4-38}{左面}{}
\shizi{4-38}{左右}{}
\shizi{4-38}{右手}{}
\shizi{4-38}{右边}{}
\shizi{4-38}{右面}{}
\shizi{4-38}{怎么}{}
\shizi{4-38}{办公}{}
\shizi{4-38}{怎么了}{}
\shizi{4-38}{怎么办}{}
}



\SecJuzi{
\shizi{4-38}{%下大雨
大雨下了一个晚上。天亮了,鸡妈妈要送小鸡宝宝上幼儿园。可是,开门一看,前后左右都是水,鸡宝宝出不了门。怎么办呢?鸡妈妈着急得很。
鹅爷爷和鸭伯伯知道了,赶快游水过来。
“快上来,快上来,爬到我们的背上来。我们驮你们去幼儿园。”
“幼儿园到啦!幼儿园到啦!”鸡宝宝一边下来,一边说。
“下午放学时,要是还是这么大的水,我们就再来驮你们回家。”鹅爷爷和鸭伯伯说。“谢谢鹅爷爷,谢谢鸭伯伯。再见!”}{}
}

\SecDanzi{
\shizi{4-39}{付}{}
\shizi{4-39}{顶}{}
\shizi{4-39}{角}{}
\shizi{4-39}{死}{}
\shizi{4-39}{扑}{}
\shizi{4-39}{期}{}
\shizi{4-39}{带}{}
\shizi{4-39}{分}{}
}

%宝宝读词语
\SecCi{
\shizi{4-39}{}{}
\shizi{4-39}{}{}
}


\SecJuzi{
\shizi{4-39}{% 三只羊
}{}
}


\SecDanzi{
\shizi{4-40}{清}{}
\shizi{4-40}{今}{}
\shizi{4-40}{昨}{}
\shizi{4-40}{光}{}
\shizi{4-40}{阴}{}
\shizi{4-40}{错}{}
\shizi{4-40}{指}{}
\shizi{4-40}{拇}{}
}

%宝宝读词语
\SecCi{
\shizi{4-40}{对付}{}
\shizi{4-40}{付给}{}
\shizi{4-40}{付出}{}
\shizi{4-40}{头顶}{}
\shizi{4-40}{顶着}{}
\shizi{4-40}{顶快}{}
\shizi{4-40}{顶大}{}
\shizi{4-40}{顶好}{}
\shizi{4-40}{顶用}{}
\shizi{4-40}{顶风}{}
\shizi{4-40}{顶牛}{}
\shizi{4-40}{顶头}{}
\shizi{4-40}{一顶}{}
\shizi{4-40}{牛角}{}
\shizi{4-40}{羊角}{}
\shizi{4-40}{lu角}{}
\shizi{4-40}{尖角}{}
\shizi{4-40}{三角}{}
\shizi{4-40}{口角}{}
\shizi{4-40}{角落}{}
\shizi{4-40}{死了}{}
\shizi{4-40}{死掉}{}
\shizi{4-40}{死人}{}
\shizi{4-40}{打死}{}
\shizi{4-40}{死掉}{}
\shizi{4-40}{死路}{}
\shizi{4-40}{死心}{}
\shizi{4-40}{高兴死了}{}
\shizi{4-40}{笑死了}{}
\shizi{4-40}{死心眼}{}
\shizi{4-40}{扑上去}{}
\shizi{4-40}{扑向前}{}
\shizi{4-40}{扑救}{}
\shizi{4-40}{日期}{}
\shizi{4-40}{星期}{}
\shizi{4-40}{学期}{}
\shizi{4-40}{定期}{}
\shizi{4-40}{长期}{}
\shizi{4-40}{分期}{}
\shizi{4-40}{到期}{}
\shizi{4-40}{带着}{}
\shizi{4-40}{带子}{}
\shizi{4-40}{带上}{}
\shizi{4-40}{带动}{}
\shizi{4-40}{带鱼}{}
\shizi{4-40}{带路}{}
\shizi{4-40}{分开}{}
\shizi{4-40}{分工}{}
\shizi{4-40}{分家}{}
\shizi{4-40}{分手}{}
\shizi{4-40}{分心}{}
}


\SecJuzi{
\shizi{4-40}{一个星期有七天:星期一、星期二、星期三、星期四、星期五、星期六,还有星期天。
星期一到星期五的五天,爸爸和妈妈都去上班,我去幼儿园。星期六和星期天,爸爸和妈妈都放假,就带我去上课。下课后,爸爸和妈妈有时带我去爷爷、奶奶家,有时带我去公园。
一个星期的七天中,我都很开心。
}{}
\shizi{4-40}{两只小手,十个指头。手心向前,手背向后。手心向西,手背向东。手心向左,手背向右。}{}
\shizi{4-40}{}{}
}


%宝宝读词语
\SecCi{
\shizi{4-复习6}{清早}{}
\shizi{4-复习6}{清醒}{}
\shizi{4-复习6}{清风}{}
\shizi{4-复习6}{清水}{}
\shizi{4-复习6}{清香}{}
\shizi{4-复习6}{分清}{}
\shizi{4-复习6}{今天}{}
\shizi{4-复习6}{今日}{}
\shizi{4-复习6}{今年}{}
\shizi{4-复习6}{今后}{}
\shizi{4-复习6}{昨天}{}
\shizi{4-复习6}{昨日}{}
\shizi{4-复习6}{阳光}{}
\shizi{4-复习6}{日光}{}
\shizi{4-复习6}{月光}{}
\shizi{4-复习6}{星光}{}
\shizi{4-复习6}{光明}{}
\shizi{4-复习6}{光头}{}
\shizi{4-复习6}{光亮}{}
\shizi{4-复习6}{眼光}{}
\shizi{4-复习6}{亮光}{}
\shizi{4-复习6}{火光}{}
\shizi{4-复习6}{光阴}{}
\shizi{4-复习6}{阴天}{}
\shizi{4-复习6}{阴云}{}
\shizi{4-复习6}{阴雨}{}
\shizi{4-复习6}{错了}{}
\shizi{4-复习6}{大错}{}
\shizi{4-复习6}{小错}{}
\shizi{4-复习6}{过错}{}
\shizi{4-复习6}{错字}{}
\shizi{4-复习6}{认错}{}
\shizi{4-复习6}{错过}{}
\shizi{4-复习6}{手指}{}
\shizi{4-复习6}{指头}{}
\shizi{4-复习6}{拇指}{}
\shizi{4-复习6}{中指}{}
\shizi{4-复习6}{小指}{}
\shizi{4-复习6}{指尖}{}
\shizi{4-复习6}{指着}{}
\shizi{4-复习6}{指向}{}
}


\SecJuzi{
\shizi{4-复习6}{}{}
\shizi{4-复习6}{}{}
}



\end{document}
