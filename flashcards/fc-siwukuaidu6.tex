%# -*- coding: utf-8 -*-
%!TEX encoding = UTF-8 Unicode
%!TEX TS-program = xelatex
% vim:ts=4:sw=4
%
% 以上设定默认使用 XeLaTex 编译,并指定 Unicode 编码,供 TeXShop 自动识别
\documentclass[avery5371,grid]{flashcards}


\cardfrontstyle[\large\slshape]{headings}
\cardbackstyle{empty}
%\cardfrontfoot{中文}
\cardfrontfoot{}%四五快读6

\newcommand{\doctitle}{四五快读 6}
\newcommand{\docauthor}{Yunhui Fu}
\newcommand{\dockeywords}{{中文}{四五快读}}
\newcommand{\docsubject}{}

% debug
\newcommand{\SecDanzi}[1]{#1} % 单字
\newcommand{\SecCi}[1]{#1} % 词
\newcommand{\SecJuzi}[1]{#1} % 句子

%\renewcommand{\SecDanzi}[1]{} % 单字
%\renewcommand{\SecCi}[1]{} % 词
%\newcommand{\comments}[1]{} % 句子


%# -*- coding: utf-8 -*-
%!TEX encoding = UTF-8 Unicode
%!TEX TS-program = xelatex
% vim:ts=4:sw=4
%
% 以上设定默认使用 XeLaTex 编译,并指定 Unicode 编码,供 TeXShop 自动识别


\newcommand\mymainfont{Noto Serif}%{Times New Roman} %{DejaVu Serif}
\newcommand\mymonofont{DejaVu Sans Mono}%{FreeMono} %{WenQuanYi Micro Hei Mono} %{Monaco}
\newcommand\myboldfont{DejaVu Sans Mono} %{WenQuanYi Micro Hei Mono}%{AR PL UKai CN}%{YaHei Consolas Hybrid}%{黑体}%{標楷體}
\newcommand\mysansfont{DejaVu Sans}%{FreeSans}
\newcommand\myitalicfont{DejaVu Serif}%{Times New Roman} %{Garamond}

\newcommand\mycjkboldfont{Microsoft YaHei} %{WenQuanYi Micro Hei Mono}%{Adobe Heiti Std}%{AR PL UKai CN}%{YaHei Consolas Hybrid}%{黑体}%{標楷體}
\newcommand\mycjkitalicfont{全字庫正楷體} %{Adobe Kaiti Std}
\newcommand\mycjkmainfont{WenyueType GutiFangsong (Noncommercial Use)}%{全字庫正楷體} %{Adobe Song Std} %{AR PL UMing CN}%{仿宋}%{宋体}%{新宋体}%{文鼎PL新宋}%
\newcommand\mycjksansfont{MingLiU} %{Adobe Ming Std}
\newcommand\mycjkmonofont{DejaVu Sans YuanTi Mono}%{WenQuanYi Micro Hei Mono}%{AR PL UMing CN}%{WenQuanYi Micro Hei Mono}


%\usepackage[nofonts]{ctex} %adobefonts
\usepackage[adobefonts]{ctex} %adobefonts
%\usepackage[fallback]{xeCJK}


\newCJKfontfamily{\mykaiti}{全字庫正楷體}%{AR PL UKai TW}%{全字庫正楷體}
\newCJKfontfamily{\myfangsong}{WenyueType GutiFangsong (Noncommercial Use)}



\usepackage{ifthen}
\usepackage{ifpdf}
\usepackage{ifxetex}
\usepackage{ifluatex}


\usepackage{color}
\usepackage[rgb]{xcolor}



\xeCJKsetup{AutoFallBack = true} % you have to use this, since fallback won't work as the xeCJK option after the ctex

\PassOptionsToPackage{
    BoldFont,  % 允許粗體
    SlantFont, % 允許斜體
    CJKnumber,
    CJKtextspaces,
    }{xeCJK}
\defaultfontfeatures{Mapping=tex-text} % 如果沒有它,會有一些 tex 特殊字符無法正常使用,比如連字符。

\xeCJKsetup {
    CheckSingle = true,
    AutoFakeBold = false,
AutoFakeSlant = false,
    CJKecglue = {},
    PunctStyle = kaiming,
KaiMingPunct+ = {:;},
}

\PassOptionsToPackage{CJKchecksingle}{xeCJK}
%\defaultCJKfontfeatures{Scale=0.5}
%\LoadClass[c5size,openany,nofonts]{ctexbook}
\XeTeXlinebreaklocale "zh"                      % 重要,使得中文可以正確斷行!
\XeTeXlinebreakskip = 0pt plus 1pt minus 0.1pt  % 给予TeX断行一定自由度
\linespread{1.3}                                % 1.3倍行距

\setCJKmainfont[BoldFont=\mycjkboldfont, ItalicFont=\mycjkitalicfont]{\mycjkmainfont}
\setCJKsansfont{\mycjksansfont}%{Adobe Ming Std} %{AR PL UMing CN} %{Microsoft YaHei}
\setCJKmonofont{\mycjkmonofont}



\ifxetex % xelatex
\else
    %The cmap package is intended to make the PDF files generated by pdflatex "searchable and copyable" in acrobat reader and other compliant PDF viewers.
    \usepackage{cmap}%
\fi
% ============================================
% Check for PDFLaTeX/LaTeX
% ============================================
\newcommand{\outengine}{xetex}
\newif\ifpdf
\ifx\pdfoutput\undefined
  \pdffalse % we are not running PDFLaTeX
  \ifxetex
    \renewcommand{\outengine}{xetex}
  \else
    \renewcommand{\outengine}{dvipdfmx}
  \fi
\else
  \pdfoutput=1 % we are running PDFLaTeX
  \pdftrue
  \usepackage{thumbpdf}
  \renewcommand{\outengine}{pdftex}
  \pdfcompresslevel=9
\fi
\usepackage[\outengine,
    bookmarksnumbered, %dvipdfmx
    %% unicode, %% 不能有unicode选项,否则bookmark会是乱码
    colorlinks=true,
    citecolor=red,
    urlcolor=blue,        % \href{...}{...} external (URL)
    filecolor=red,      % \href{...} local file
    linkcolor=black, % \ref{...} and \pageref{...}
    breaklinks,
    pdftitle={\doctitle},
    pdfauthor={\docauthor},
    pdfsubject={\docsubject},
    pdfkeywords={\dockeywords},
    pdfproducer={Latex with hyperref},
    pdfcreator={pdflatex},
    %%pdfadjustspacing=1,
    pdfborder=1,
    pdfpagemode=UseNone,
    pagebackref,
    bookmarksopen=true]{hyperref}

% --------------------------------------------
% Load graphicx package with pdf if needed 
% --------------------------------------------
\ifxetex    % xelatex
    \usepackage{graphicx}
\else
    \ifpdf
        \usepackage[pdftex]{graphicx}
        \pdfcompresslevel=9
    \else
        \usepackage{graphicx} % \usepackage[dvipdfm]{graphicx}
    \fi
\fi

%\newCJKfontfamily{\yanti}{\mycjkmainfont}
%\newenvironment{shicibody}{%
%\begin{verse}\centering\yanti\large\hspace{13pt}}{\end{verse}}


\usepackage{xstring}

%The baseline-skip should be set to roughly 1.2x the font size.
%\fontsize{size}{baselineskip}
%\fontsize{50}{60}

% 计算所能容纳的字数
\newcommand\shizishow[1]{
\StrLen{#1}[\mystringlen]
%\def\mythresh{1}
    \ifthenelse{\mystringlen < 2}{
        {\fontsize{130}{156} #1}
    }{ \ifthenelse{\mystringlen < 3}{
        {\fontsize{110}{122} #1}%{\fontsize{100}{120} #1}
    }{ \ifthenelse{\mystringlen < 4}{
        {\fontsize{78}{94} #1}%{\fontsize{90}{108} #1}
    }{ \ifthenelse{\mystringlen < 5}{
        {\fontsize{55}{66} #1}
    }{ \ifthenelse{\mystringlen < 7}{
        {\fontsize{70}{84} #1}
    }{ \ifthenelse{\mystringlen < 9}{
        {\fontsize{55}{66} #1}
    }{ \ifthenelse{\mystringlen < 16}{
        {\zihao{0} #1} %5x4=20
    }{ \ifthenelse{\mystringlen < 49}{
        {\zihao{1} #1} % 8x6=48
    }{ \ifthenelse{\mystringlen < 55}{
        {\Huge #1} % 9x6=54
    }{ \ifthenelse{\mystringlen < 61}{
        {\zihao{2} #1} % 10x6=60
    }{ \ifthenelse{\mystringlen < 78}{
        {\huge #1} % 11x7=77
    }{ \ifthenelse{\mystringlen < 105}{
        {\LARGE #1} % 13x8=104
    }{ \ifthenelse{\mystringlen < 113}{
        {\zihao{3} #1} % 14x8=112
    }{ \ifthenelse{\mystringlen < 129}{
        {\Large #1} % 16x8=128 %{\zihao{4} #1} % 16x8=128
    }{ \ifthenelse{\mystringlen < 153}{
        {\large #1} % 19x8=152, or 19x9=171
    }{ \ifthenelse{\mystringlen < 198}{
        {\zihao{5} #1} % 22x9=198
    }{ \ifthenelse{\mystringlen < 207}{
        {\normalsize #1} % 23x9=207
    }{ \ifthenelse{\mystringlen < 234}{
        {\small #1} % 26x9=234
    }{ \ifthenelse{\mystringlen < 261}{
        {\footnotesize #1} % 29x9=261
    }{ \ifthenelse{\mystringlen < 279}{
        {\zihao{6} #1} % 31x9=279
    }{ \ifthenelse{\mystringlen < 297}{
        {\scriptsize #1} % 33x9=297
    }{ \ifthenelse{\mystringlen < 378}{
        {\zihao{7} #1} % 42x9=378
    }{
        {\tiny #1} % 46x9=414 %{\zihao{8} #1} % 46x9=414
    }}}}}}}}}}}}}}}}}}}}}}
}

% 计算包含拼音时所能容纳的字数
\newcommand\shizipy[1]{
\StrLen{#1}[\mystringlen]
\xpinyin*{
    \ifthenelse{\mystringlen < 6}{
        {\zihao{0} #1} %5x4=20
    }{ \ifthenelse{\mystringlen < 18}{
        {\zihao{1} #1} % 8x6=48
    }{ \ifthenelse{\mystringlen < 22}{
        {\Huge #1} % 9x6=54
    }{ \ifthenelse{\mystringlen < 31}{
        {\zihao{2} #1} % 10x6=60
    }{ \ifthenelse{\mystringlen < 34}{
        {\huge #1} % 11x7=77
    }{ \ifthenelse{\mystringlen < 53}{
        {\LARGE #1} % 13x8=104
    }{ \ifthenelse{\mystringlen < 57}{
        {\zihao{3} #1} % 14x8=112
    }{ \ifthenelse{\mystringlen < 65}{
        {\Large #1} % 16x8=128 %{\zihao{4} #1} % 16x8=128
    }{ \ifthenelse{\mystringlen < 77}{
        {\large #1} % 19x8=152, or 19x9=171
    }{ \ifthenelse{\mystringlen < 89}{
        {\zihao{5} #1} % 22x9=198
    }{ \ifthenelse{\mystringlen < 93}{
        {\normalsize #1} % 23x9=207
    }{ \ifthenelse{\mystringlen < 105}{
        {\small #1} % 26x9=234
    }{ \ifthenelse{\mystringlen < 117}{
        {\footnotesize #1} % 29x9=261
    }{ \ifthenelse{\mystringlen < 125}{
        {\zihao{6} #1} % 31x9=279
    }{ \ifthenelse{\mystringlen < 133}{
        {\scriptsize #1} % 33x9=297
    }{ \ifthenelse{\mystringlen < 169}{
        {\zihao{7} #1} % 42x9=378
    }{
        {\tiny #1} % 46x9=414 %{\zihao{8} #1} % 46x9=414
    }}}}}}}}}}}}}}}}
}
}

% 计算在comment中(三字经)能容纳的字数
\newcommand\sanzicomments[1]{
\StrLen{#1}[\mystringlen]
%\def\mythresh{1}
    \ifthenelse{\mystringlen < 7}{
        {\zihao{0} #1} %5x4=20
    }{ \ifthenelse{\mystringlen < 22}{
        {\zihao{1} #1} % 8x6=48
    }{ \ifthenelse{\mystringlen < 25}{
        {\Huge #1} % 9x6=54
    }{ \ifthenelse{\mystringlen < 38}{
        {\zihao{2} #1} % 10x6=60
    }{ \ifthenelse{\mystringlen < 42}{ %
        {\huge #1} % 11x7=77
    }{ \ifthenelse{\mystringlen < 63}{
        {\LARGE #1} % 13x8=104
    }{ \ifthenelse{\mystringlen < 64}{
        {\zihao{3} #1} % 14x8=112
    }{ \ifthenelse{\mystringlen < 87}{
        {\Large #1} % 16x8=128 %{\zihao{4} #1} % 16x8=128
    }{ \ifthenelse{\mystringlen < 125}{
        {\large #1} % 19x8=152, or 19x9=171
    }{ \ifthenelse{\mystringlen < 173}{
        {\zihao{5} #1} % 22x9=198
    }{ \ifthenelse{\mystringlen < 205}{
        {\normalsize #1} % 23x9=207
    }{ \ifthenelse{\mystringlen < 232}{
        {\small #1} % 26x9=234
    }{ \ifthenelse{\mystringlen < 259}{
        {\footnotesize #1} % 29x9=261
    }{ \ifthenelse{\mystringlen < 277}{
        {\zihao{6} #1} % 31x9=279
    }{ \ifthenelse{\mystringlen < 295}{
        {\scriptsize #1} % 33x9=297
    }{ \ifthenelse{\mystringlen < 376}{
        {\zihao{7} #1} % 42x9=378
    }{
        {\tiny #1} % 46x9=414 %{\zihao{8} #1} % 46x9=414
    }}}}}}}}}}}}}}}}
}



% 诗词
\newcommand\shici[5]{
\begin{flashcard}[#1 -- #2 (#3) #4]{%
{\mykaiti \Large {#5}} %\\
}
\vspace*{\stretch{1}}
\begin{center}

{\mykaiti \zihao{1} {#2}}\\ \vspace*{\stretch{.5}}
{\large(#3)}\\ \vspace*{\stretch{.25}}
{\LARGE #4}

\end{center}
\vspace*{\stretch{1}}
\end{flashcard}
}



% 识字

%\usepackage[overlap,CJK]{ruby}
\usepackage{xpinyin}

% 如果字数多,则略写
\newcommand\shizititle[2]{
  \StrLen{(#1) #2}[\mystringlen]%
  \ifthenelse{\mystringlen > 24}{
    \StrLeft{(#1) #2}{20}......\StrRight{(#1) #2}{4}
  }{
  (#1) #2
  }
}
\newcommand\shizi[3]{
\begin{flashcard}[\shizititle{#1}{#2}]{ %
{\mykaiti \shizipy{#2}}

{\myfangsong \normalsize #3} %
} %
\vspace*{\stretch{1}}
\begin{center}
%ēéěè
%\ruby{莉}{li}
%\xpinyin*{床前明月光,疑是地上霜。举头望明月,低头思故乡。}
%\begin{pinyinscope}
%床前明月光,疑是地上霜。举头望明月,低头思故乡。
%\end{pinyinscope}

{\mykaiti \shizishow{#2}}

\end{center}
\vspace*{\stretch{1}}
\end{flashcard}
}




% 三字经
\newcommand\sanzijing[3]{
\begin{flashcard}[]{\mykaiti \fontsize{26}{\baselineskip}\selectfont \xpinyin*{#1}}%{\Huge \xpinyin*{#1}}
\vspace*{\stretch{1}}
%\begin{center}
%\normalsize
{\myfangsong
\sanzicomments{
【解释】 #2

〖解读〗 #3
}}
%\end{center}
\vspace*{\stretch{1}}
\end{flashcard}
}





\newcommand\docshowcopyright{
\begin{flashcard}[Copyright \& License]{Copyright \copyright \, 2014 Yunhui Fu\\
Some rights reserved.}
\vspace*{\stretch{1}}
These flashcards and the accompanying \LaTeX \, source code are licensed
under a Creative Commons Attribution--NonCommercial--ShareAlike 2.5 License.  
For more information, see creativecommons.org.  You can contact the author at:
\begin{center}
\begin{small}\tt yhfudev at gmail com\end{small}

\medskip
File last updated on \today, \\
at \currenttime
\end{center}
\vspace*{\stretch{1}}
\end{flashcard}

\begin{flashcard}[版权申明]{版权所有 \copyright \, 2014 Yunhui Fu\\
有些版权保留.}
\vspace*{\stretch{1}}
闪卡及其 \LaTeX \, 源代码在
署名-相同方式共享 2.5 下保护.
参见 creativecommons.org.  你可以联系作者:
\begin{center}
\begin{small}\tt yhfudev at gmail com\end{small}

\medskip
文件最近更新: \today, \\
\currenttime
\end{center}
\vspace*{\stretch{1}}
\end{flashcard}
}






\newcommand\docshowtitle[3]{
\begin{flashcard}[]{ %
{\mykaiti \Huge{#1}}

{\myfangsong \normalsize #2} %
} %
\vspace*{\stretch{1}}
\begin{center}

{\mykaiti \shizishow{#3}}

\end{center}
\vspace*{\stretch{1}}
\end{flashcard}
}








\usepackage{datetime}



\begin{document}

\docshowcopyright
\docshowtitle{\doctitle}{\docauthor}{%
使用双面打印,然后按线剪下。
}


\SecDanzi{
\shizi{6-51}{停}{}
\shizi{6-51}{而}{}
\shizi{6-51}{窗}{}
\shizi{6-51}{刚}{}
\shizi{6-51}{撞}{}
\shizi{6-51}{比}{}
\shizi{6-51}{记}{}
\shizi{6-51}{串}{}
\shizi{6-51}{被}{}
\shizi{6-51}{够}{}
\shizi{6-51}{命}{}
\shizi{6-51}{颈}{}
\shizi{6-51}{定}{}
\shizi{6-51}{吓}{}
\shizi{6-51}{啄}{}
\shizi{6-51}{破}{}
}

% 宝宝读词语
\SecCi{
\shizi{6-51}{停车}{}
\shizi{6-51}{停火}{}
\shizi{6-51}{停住}{}
\shizi{6-51}{停了}{}
\shizi{6-51}{窗口}{}
\shizi{6-51}{天窗}{}
\shizi{6-51}{停放}{}
\shizi{6-51}{停水}{}
\shizi{6-51}{停不停}{}
\shizi{6-51}{而今}{}
\shizi{6-51}{窗子}{}
\shizi{6-51}{开窗}{}
\shizi{6-51}{停工}{}
\shizi{6-51}{停学}{}
\shizi{6-51}{不停}{}
\shizi{6-51}{窗洞}{}
\shizi{6-51}{窗花}{}
\shizi{6-51}{关窗}{}
\shizi{6-51}{门窗}{}
\shizi{6-51}{刚才}{}
\shizi{6-51}{刚刚}{}
\shizi{6-51}{刚好}{}
\shizi{6-51}{刚走}{}
\shizi{6-51}{撞见}{}
\shizi{6-51}{不比}{}
\shizi{6-51}{记得}{}
\shizi{6-51}{日记}{}
\shizi{6-51}{刚正}{}
\shizi{6-51}{刚停}{}
\shizi{6-51}{撞上}{}
\shizi{6-51}{记住}{}
\shizi{6-51}{不记得}{}
\shizi{6-51}{笔记}{}
\shizi{6-51}{刚来}{}
\shizi{6-51}{撞车}{}
\shizi{6-51}{比一比}{}
\shizi{6-51}{记不住}{}
\shizi{6-51}{记不清}{}
\shizi{6-51}{游记}{}
\shizi{6-51}{一串串}{}
\shizi{6-51}{串门}{}
}

% 宝宝读短文
\SecJuzi{
\shizi{6-51}{%交通指示灯
星期天,爸爸开着汽车带着我和朝阳哥哥一起去森林公园,因为是星期天,马路上的汽车很多,所以车开得很慢。爸爸一路上问了我们很多问题。
爸爸问:“知道绿灯亮是应该走,还是应该停?”我和朝阳哥哥一起喊着说:“应该走。”
“那么红灯亮呢?”
“应该停。”
“黄灯亮呢?”
“不知道。”
“你们看到了没有?红灯亮了以后,黄灯亮。黄灯亮了以后,绿灯亮。红灯不能走,绿灯能走。而黄灯是在能走和不能走中间时才亮。好好想一想,看你们两个谁聪明。”
“啊!我明白了。是停车以后,告诉你快要走时,才亮黄灯。对吗?”
“对啦!小朝阳真聪明。”爸爸摸了摸朝阳哥哥的头。
正在这时,听到“呜——”救火车的声音。我马上把头伸到了窗外,想看个明白。朝阳哥哥一把把我拉了回来。“在开车时,不能把头和手伸到车窗外边。要是刚好有车从边上开过去,会把你的头和手撞掉。多可怕!”
“还是朝阳哥哥比你知道的事多。你要好好记住。知道吗?”}{}
}


\SecDanzi{
\shizi{6-52}{湖}{}
\shizi{6-52}{棵}{}
\shizi{6-52}{瓜}{}
\shizi{6-52}{透}{}
\shizi{6-52}{腿}{}
\shizi{6-52}{狐}{}
\shizi{6-52}{粗}{}
\shizi{6-52}{竿}{}
\shizi{6-52}{狸}{}
\shizi{6-52}{跟}{}
}

% 宝宝读词语
\SecCi{
\shizi{6-52}{被子}{}
\shizi{6-52}{不够}{}
\shizi{6-52}{坐定}{}
\shizi{6-52}{能够}{}
\shizi{6-52}{够着了}{}
\shizi{6-52}{生命}{}
\shizi{6-52}{定心}{}
\shizi{6-52}{不一定}{}
\shizi{6-52}{啄破}{}
\shizi{6-52}{被告}{}
\shizi{6-52}{够得着}{}
\shizi{6-52}{够朋友}{}
\shizi{6-52}{吓人}{}
\shizi{6-52}{定时}{}
\shizi{6-52}{救命}{}
\shizi{6-52}{够不够}{}
\shizi{6-52}{长颈鹿}{}
\shizi{6-52}{定期}{}
\shizi{6-52}{啄米}{}
\shizi{6-52}{人命}{}
\shizi{6-52}{啄木鸟}{}
\shizi{6-52}{破了}{}
\shizi{6-52}{够不着}{}
\shizi{6-52}{一定}{}
\shizi{6-52}{啄食}{}
\shizi{6-52}{破开}{}
}

% 宝宝读短文
\SecJuzi{
\shizi{6-52}{%小白兔上天
兔妈妈拿着一串气球给小白兔,告诉他说气球是大猩猩伯伯送的,小白兔高兴地接过气球。可是,他一接过这串气球,就被气球带着,往天上飞去。
兔妈妈跳啊跳啊,根本够不着,她着急的大声喊叫:“快快救救小白兔,谁快来救救小白兔呀!”
长颈鹿听见了,赶快跑来,他想,自己这么高,一定能够拉住往上飞的小白兔。可是,长颈鹿够不到小白兔。
小猴子也听到兔妈妈的喊声,他飞快
地爬上了树,想去救小白兔。可是,他也够不到往天上飞得很快的小白兔。
被气球带往天空的小白兔吓得一边“呜——”地哭,一边喊着:“救命啊!救命啊!谁来救救我呀!”
正带着小鸟在天上练习飞的鸟妈妈听见了小白兔的哭喊声,马上对小鸟们说:“我们赶快去救小白兔!”
“我们怎么救他呀?”小鸟问。
“我们去啄气球,把气球啄破了,小白兔就可以掉到地上去了。可是不能把气球一下都啄破,要一个一个地啄破。”鸟妈妈说。
“为什么不能一下子都啄破呢?”
“要是一下子把气球都啄破,小白兔就会很快往下掉,那样会摔死的。一个一个地啄破,小白兔慢慢地往地上掉,就没有事了。”
小鸟们飞到小白兔身边,一个、一个地啄破气球,当啄破两个气球以后,小白兔就停住了,不再往上飞。当啄破四个气球以后,小白兔开始往下掉。
就这样,小白兔慢慢慢慢地掉到了地上。
兔妈妈扑到小白兔身边,抱起小白兔,一边亲他,一边说:“你真是个勇敢的孩子。”又一边对救了他的小鸟们说:“谢谢你们,谢谢你们救了我的孩子。”}{}
}



\SecDanzi{
\shizi{6-53}{钓}{}
\shizi{6-53}{甩}{}
\shizi{6-53}{钩}{}
\shizi{6-53}{忘}{}
\shizi{6-53}{装}{}
\shizi{6-53}{饵}{}
\shizi{6-53}{算}{}
\shizi{6-53}{桶}{}
\shizi{6-53}{坏}{}
\shizi{6-53}{忙}{}
}

% 宝宝读词语
\SecCi{
\shizi{6-53}{湖水}{}
\shizi{6-53}{南瓜}{}
\shizi{6-53}{瓜果}{}
\shizi{6-53}{透雨}{}
\shizi{6-53}{透过}{}
\shizi{6-53}{狐狸}{}
\shizi{6-53}{一棵}{}
\shizi{6-53}{西瓜}{}
\shizi{6-53}{瓜子}{}
\shizi{6-53}{透亮}{}
\shizi{6-53}{大腿}{}
\shizi{6-53}{狸猫}{}
\shizi{6-53}{冬瓜}{}
\shizi{6-53}{木瓜}{}
\shizi{6-53}{透风}{}
\shizi{6-53}{看透}{}
\shizi{6-53}{狗腿}{}
\shizi{6-53}{跟着}{}
\shizi{6-53}{两棵}{}
\shizi{6-53}{黄瓜}{}
\shizi{6-53}{瓜分}{}
\shizi{6-53}{透明}{}
\shizi{6-53}{小腿}{}
\shizi{6-53}{跟前}{}
\shizi{6-53}{跟头}{}
\shizi{6-53}{粗心}{}
\shizi{6-53}{粗细}{}
\shizi{6-53}{竹竿}{}
}

% 宝宝读短文
\SecJuzi{
\shizi{6-53}{%“咕咚”来了
湖边有棵木瓜树,树边住着小白兔。一天,有只熟透了的木瓜,被风一吹,从树上落下来,“咕咚”一声,正好掉在湖里。
小白兔听到“咕咚”一声,吓得拔腿就跑。狐狸看见小白兔跑得那么快,就问:“除了什么事?”小白兔一边跑一边说:“咕咚——,咕咚——!”
狐狸看到小白兔那害怕的样子,以为“咕咚”是个很可怕的东西,就也跟着跑起来。
猴子看到他们没命地跑,就赶上去问:
“出了什么事?”狐狸说:“‘咕咚’来了!” 猴子想,狐狸吓成这个样子,“咕咚”一定是个很可怕的东西,就也跟着跑起来。
路上,他们又碰到狗熊、梅花鹿、老虎。
他们同样问:“出了什么事?”
“‘咕咚’来了!”于是,大家也都跟着没命地跑起来。
最后,他们碰到了一只长毛狮子,他喊住了大家,问:“什么东西把你们吓成这个样子?”
大家上气不接下气地说:“不得了,‘咕咚’来了!”
长毛狮子问:“‘咕咚’是谁?它在哪里呀?”大家都说不知道。
小白兔说:“那个‘咕咚’就在我住的湖边。”
长毛狮子说:“那好,你带我们去看看。”
到了湖边,大家东找找,西看看,哪里有什么“咕咚”呀!正在这时,又有一只熟透了的木瓜掉到湖里,有是“咕咚”一声。
这时,大家才明白,“咕咚”原来是木瓜掉水里的声音。}{}
}



\SecDanzi{
\shizi{6-54}{洗}{}
\shizi{6-54}{忽}{}
\shizi{6-54}{然}{}
\shizi{6-54}{全}{}
\shizi{6-54}{条}{}
\shizi{6-54}{怪}{}
\shizi{6-54}{物}{}
\shizi{6-54}{猪}{}
\shizi{6-54}{影}{}
\shizi{6-54}{信}{}
}


% 宝宝读词语
\SecCi{
\shizi{6-54}{甩开}{}
\shizi{6-54}{甩掉}{}
\shizi{6-54}{忘掉}{}
\shizi{6-54}{西装}{}
\shizi{6-54}{鱼饵}{}
\shizi{6-54}{算数}{}
\shizi{6-54}{甩手}{}
\shizi{6-54}{甩去}{}
\shizi{6-54}{钩子}{}
\shizi{6-54}{忘我}{}
\shizi{6-54}{装好}{}
\shizi{6-54}{算了}{}
\shizi{6-54}{钓鱼}{}
\shizi{6-54}{甩出}{}
\shizi{6-54}{鱼钩}{}
\shizi{6-54}{忘记}{}
\shizi{6-54}{安装}{}
\shizi{6-54}{钓饵}{}
\shizi{6-54}{心算}{}
\shizi{6-54}{水桶}{}
\shizi{6-54}{坏人}{}
\shizi{6-54}{乐坏了}{}
\shizi{6-54}{急忙}{}
\shizi{6-54}{算题}{}
\shizi{6-54}{桶子}{}
\shizi{6-54}{坏了}{}
\shizi{6-54}{气坏了}{}
\shizi{6-54}{很忙}{}
\shizi{6-54}{总算}{}
\shizi{6-54}{木桶}{}
\shizi{6-54}{坏事}{}
\shizi{6-54}{忙着}{}
}



% 宝宝读短文
\SecJuzi{
\shizi{6-54}{%粗心的小猫
星期天大清早,小猫高高兴兴地背上鱼竿去钓鱼。
到了河边,小猫把鱼竿甩到河里等着鱼上钩。等呀等,太阳老高了,还没有鱼上钩。呀!原来忘了装鱼钩。
小猫装好了鱼钩,把它甩到河里。等啊等,太阳快要下山了,还是没有鱼上钩。呀!原来又忘了装鱼饵。
小猫又把鱼饵装到鱼钩上,把它放进水里。等啊等,太阳快要落山时,总算钓上一条好大好大的鱼。
小猫高兴得马上想拿回家给妈妈。坏了!忘记带装鱼的桶子。小猫只好用双手抱着鱼,急急忙忙跑回去。
跑着跑着,小猫一下想起拿回大鱼竿了。只好再回去拿鱼竿。
鱼竿还在河边。小猫去拿鱼竿时,手里的大鱼一跳,落到地上,再一跳跳到河里,一游就不见了。
粗心的小猫一条鱼也没有钓到,只好背着大鱼竿回家去了。}{}
}


\SecDanzi{
\shizi{6-55}{呼}{}
\shizi{6-55}{结}{}
\shizi{6-55}{冰}{}
\shizi{6-55}{枝}{}
\shizi{6-55}{发}{}
\shizi{6-55}{抖}{}
\shizi{6-55}{暖}{}
\shizi{6-55}{屋}{}
\shizi{6-55}{法}{}
\shizi{6-55}{件}{}
\shizi{6-55}{于}{}
}


% 宝宝读词语
\SecCi{
\shizi{6-55}{洗手}{}
\shizi{6-55}{忽然}{}
\shizi{6-55}{全家}{}
\shizi{6-55}{全面}{}
\shizi{6-55}{条子}{}
\shizi{6-55}{怪物}{}
\shizi{6-55}{洗脸}{}
\shizi{6-55}{然后}{}
\shizi{6-55}{全班}{}
\shizi{6-55}{一条路}{}
\shizi{6-55}{柳条}{}
\shizi{6-55}{怪样}{}
\shizi{6-55}{清洗}{}
\shizi{6-55}{全身}{}
\shizi{6-55}{全力}{}
\shizi{6-55}{两条鱼}{}
\shizi{6-55}{怪人}{}
\shizi{6-55}{怪不得}{}
\shizi{6-55}{动物}{}
\shizi{6-55}{小猪}{}
\shizi{6-55}{电影}{}
\shizi{6-55}{写信}{}
\shizi{6-55}{信心}{}
\shizi{6-55}{生物}{}
\shizi{6-55}{影子}{}
\shizi{6-55}{来信}{}
\shizi{6-55}{信不信}{}
\shizi{6-55}{大猪}{}
\shizi{6-55}{倒影}{}
\shizi{6-55}{去信}{}
\shizi{6-55}{不信}{}
}


% 宝宝读短文
\SecJuzi{
\shizi{6-55}{%怪物
一天,小猴子去河边洗手,忽然,它看见水里有一个全身长毛,还有一条长尾巴的怪物正看着他。这可把小猴子吓坏了。
小猴子赶快就跑,上气不接下气地跑到小猪家:“河里有个全身长毛,还有一条长尾巴的怪物,好可怕啊!”
“真的?带我去看看。”
“我不去,我怕。”
“我们一起去,就不怕了。”
他们两个来到河边,小猪往河里一看,看见一个大嘴巴、大耳朵、鼻子朝天的怪物正看着他。
“呀!快跑,真有怪物。”小猪和小猴子没命地跑。
“小猪,小猴子,你们跑那么快,是要到哪里去呀?”大象问道。
“河里有个大嘴巴、大耳朵、鼻子朝天的怪物。”
“不对,是个全身长毛,还有一条长尾巴的怪物,好可怕的。”
“啊!我知道啦。”大象笑了起来,“那不是怪物,是你们自己的倒影。不信,
你们再回去看看。”
小猪和小猴子又回到河边:“原来这就是倒影啊!”
他们两个,我看看你,你看看我,都笑了。
河里的“怪物”也笑了。}{}
}


\SecDanzi{
\shizi{6-56}{松}{}
\shizi{6-56}{帽}{}
\shizi{6-56}{戴}{}
\shizi{6-56}{热}{}
\shizi{6-56}{候}{}
\shizi{6-56}{劳}{}
\shizi{6-56}{流}{}
\shizi{6-56}{汗}{}
\shizi{6-56}{凉}{}
\shizi{6-56}{别}{}
}

% 宝宝读词语
\SecCi{
\shizi{6-56}{呼叫}{}
\shizi{6-56}{呼气}{}
\shizi{6-56}{结果}{}
\shizi{6-56}{冰花}{}
\shizi{6-56}{冰山}{}
\shizi{6-56}{呼喊}{}
\shizi{6-56}{结子}{}
\shizi{6-56}{结尾}{}
\shizi{6-56}{冰冷}{}
\shizi{6-56}{树枝}{}
\shizi{6-56}{呼声}{}
\shizi{6-56}{结冰}{}
\shizi{6-56}{冰雪}{}
\shizi{6-56}{冰水}{}
\shizi{6-56}{枝条}{}
\shizi{6-56}{发生}{}
\shizi{6-56}{发火}{}
\shizi{6-56}{发明}{}
\shizi{6-56}{发给}{}
\shizi{6-56}{抖掉}{}
\shizi{6-56}{发电}{}
\shizi{6-56}{发亮}{}
\shizi{6-56}{头发}{}
\shizi{6-56}{发抖}{}
\shizi{6-56}{冷暖}{}
\shizi{6-56}{发水}{}
\shizi{6-56}{发出}{}
\shizi{6-56}{出发}{}
\shizi{6-56}{抖动}{}
\shizi{6-56}{暖和}{}
\shizi{6-56}{暖水瓶}{}
\shizi{6-56}{屋子}{}
\shizi{6-56}{方法}{}
\shizi{6-56}{一件}{}
\shizi{6-56}{急件}{}
\shizi{6-56}{里屋}{}
\shizi{6-56}{屋顶}{}
\shizi{6-56}{法宝}{}
\shizi{6-56}{两件}{}
\shizi{6-56}{于是}{}
\shizi{6-56}{外屋}{}
\shizi{6-56}{办法}{}
\shizi{6-56}{法国}{}
\shizi{6-56}{文件}{}
\shizi{6-56}{对于}{}
\shizi{6-56}{用于}{}
}


% 宝宝读短文
\SecJuzi{
\shizi{6-56}{%北风爷爷的礼物
冬天到了,北风爷爷呼呼地唱着歌,高高兴兴地出了家门。
北风爷爷一边走,一边吹风。风吹到了小河,小河里的水就结成了冰。北风爷爷吹到小树林,树枝冷得直发抖。北风爷爷去找小鸟玩,小鸟说:“北风爷爷,你吹得我好冷呀,我要回家了。家里很暖和。”北风爷爷想找小朋友玩,小朋友们也说:“北风爷爷,你吹得我们好冷呀,我们要回家了。家里可暖和呢!”
北风爷爷生气了,谁都不和我玩,他就“咚咚”地敲门,还是想叫小朋友出来和他一起玩。小朋友说:“北风爷爷,外面好冷。我们还是在屋子里玩吧!”
北风爷爷想出了一个好办法,他大声地喊:“呼——,呼——,小朋友们,快出来!看看我给你们带来一件什么样的礼物呀!”
小朋友们把门打开一看,呀!一片,两片,三片……好美丽的雪花呀!地上白了,屋顶白了,树枝也白了。于是,一个,两个,三个……许多小朋友都从屋子里跑出来,和北风爷爷一起高高兴兴地打雪仗、堆雪人、做游戏,北风爷爷说:“你们真是不怕冷又勇敢的好孩子!”}{}
}


\SecDanzi{
\shizi{6-57}{整}{}
\shizi{6-57}{还}{}
\shizi{6-57}{觉}{}
\shizi{6-57}{盖}{}
\shizi{6-57}{怜}{}
\shizi{6-57}{窝}{}
\shizi{6-57}{着}{}
\shizi{6-57}{病}{}
\shizi{6-57}{受}{}
\shizi{6-57}{羽}{}
\shizi{6-57}{望}{}
}


% 宝宝读词语
\SecCi{
\shizi{6-57}{松树}{}
\shizi{6-57}{松口}{}
\shizi{6-57}{轻松}{}
\shizi{6-57}{太阳帽}{}
\shizi{6-57}{热带鱼}{}
\shizi{6-57}{热了}{}
\shizi{6-57}{松子}{}
\shizi{6-57}{松手}{}
\shizi{6-57}{帽子}{}
\shizi{6-57}{戴上}{}
\shizi{6-57}{热天}{}
\shizi{6-57}{热水}{}
\shizi{6-57}{松鼠}{}
\shizi{6-57}{放松}{}
\shizi{6-57}{草帽}{}
\shizi{6-57}{戴帽子}{}
\shizi{6-57}{很热}{}
\shizi{6-57}{热爱}{}
\shizi{6-57}{火热}{}
\shizi{6-57}{等候}{}
\shizi{6-57}{劳动}{}
\shizi{6-57}{冰凉}{}
\shizi{6-57}{别的}{}
\shizi{6-57}{别走}{}
\shizi{6-57}{热心}{}
\shizi{6-57}{问候}{}
\shizi{6-57}{时候}{}
\shizi{6-57}{凉快}{}
\shizi{6-57}{别人}{}
\shizi{6-57}{别给}{}
\shizi{6-57}{冷热}{}
\shizi{6-57}{候鸟}{}
\shizi{6-57}{凉水}{}
\shizi{6-57}{清凉}{}
\shizi{6-57}{个别}{}
\shizi{6-57}{别哭}{}
\shizi{6-57}{别说话}{}
\shizi{6-57}{告别}{}
}

% 宝宝读短文
\SecJuzi{
\shizi{6-57}{%会变得太阳帽
小松鼠有一顶会变大又会变小的太阳帽,戴上它,就一点也不热了。
夏天的时候,大家都在太阳下面劳动,热得流大汗。
小松鼠看见小蚂蚁很热,就把太阳帽给小蚂蚁戴上,帽子到了小蚂蚁头上就变小了。小蚂蚁戴上它,很凉快。小蚂蚁戴了一会儿,就把帽子还给小松鼠,说:“你自己戴吧,你比我更热!”
小松鼠又把太阳帽给流汗的大象戴在头上,帽子就变大了。大象戴上帽子,一点也不热了。大象戴了一会儿,也还给了小松鼠,说:“你给别人吧,别人比我更热。”
小松鼠看着大家都在太阳下面劳动,真热呀!这顶帽子给谁戴好呢?
小松鼠想啊想,想到了一个好办法。他爬到了大山的山顶,把太阳帽戴在山顶的尖尖上,太阳帽忽然变得很大很大,把整座大山都盖住了。大家在帽子下面劳动,觉得很凉快,一点汗也没有了。于是,大家一起唱快乐的歌。}{}
}


\SecDanzi{
\shizi{6-58}{骑}{}
\shizi{6-58}{行}{}
\shizi{6-58}{远}{}
\shizi{6-58}{经}{}
\shizi{6-58}{旁}{}
\shizi{6-58}{婶}{}
\shizi{6-58}{呵}{}
\shizi{6-58}{稀}{}
\shizi{6-58}{奇}{}
}

% 宝宝读词语
\SecCi{
\shizi{6-58}{整个}{}
\shizi{6-58}{还书}{}
\shizi{6-58}{送还}{}
\shizi{6-58}{盖着}{}
\shizi{6-58}{鸟窝}{}
\shizi{6-58}{着急}{}
\shizi{6-58}{整理}{}
\shizi{6-58}{还手}{}
\shizi{6-58}{觉得}{}
\shizi{6-58}{可怜}{}
\shizi{6-58}{心窝}{}
\shizi{6-58}{着火}{}
\shizi{6-58}{还给}{}
\shizi{6-58}{还东西}{}
\shizi{6-58}{盖子}{}
\shizi{6-58}{怜爱}{}
\shizi{6-58}{窝头}{}
\shizi{6-58}{睡着了}{}
\shizi{6-58}{病人}{}
\shizi{6-58}{病倒}{}
\shizi{6-58}{受害}{}
\shizi{6-58}{接受}{}
\shizi{6-58}{望见}{}
\shizi{6-58}{生病}{}
\shizi{6-58}{大病}{}
\shizi{6-58}{受到}{}
\shizi{6-58}{羽毛}{}
\shizi{6-58}{看望}{}
\shizi{6-58}{看病}{}
\shizi{6-58}{小病}{}
\shizi{6-58}{难受}{}
\shizi{6-58}{望着}{}
}

% 宝宝读短文
\SecJuzi{
\shizi{6-58}{%帽子鸟窝
冬天到了,北风呼呼地吹,天气很冷,有一只小鸟真可怜,它在树枝上直发抖。
一位老爷爷走过来,他看见小鸟在树枝上发抖,就问小鸟:“你为什么不回家?在这里发抖?”
小鸟说:“风把我们的鸟窝吹走了,我们没有家了。”
老爷爷说:“别着急,我来帮你想办法。”老爷爷就用自己的帽子给小鸟做了鸟窝。老爷爷的帽子可真暖和!
小鸟想,树林里还有很多怕冷的小鸟,一定也在发抖,快把他们也叫来吧。
于是,小鸟们都飞进了老爷爷的帽子里,真暖和呀!他们高兴地唱着歌:“小鸟飞来,小鸟飞来,爷爷的帽子真暖和。谢谢爷爷,谢谢爷爷,住在这里不冷了。”
以后,老爷爷天天都来看望小鸟,听小鸟们唱歌,大家都很开心。
可是,有一天,老爷爷没有来。原来老爷爷病了。
小鸟们想,一定是老爷爷把帽子给了我们,他自己受凉生了病。我们来给他做一顶帽子吧。
于是,小鸟们就用自己的羽毛做成了
一顶帽子送给了老爷爷。
过了几天,老爷爷的病好了,他又来看望小鸟,小鸟们又唱起了快乐的歌。}{}
}


\SecDanzi{
\shizi{6-59}{祝}{}
\shizi{6-59}{贺}{}
\shizi{6-59}{哟}{}
\shizi{6-59}{脑}{}
\shizi{6-59}{呆}{}
\shizi{6-59}{床}{}
\shizi{6-59}{闹}{}
\shizi{6-59}{钟}{}
\shizi{6-59}{拨}{}
\shizi{6-59}{准}{}
\shizi{6-59}{备}{}
}

% 宝宝读词语
\SecCi{
\shizi{6-59}{骑车}{}
\shizi{6-59}{行人}{}
\shizi{6-59}{很远}{}
\shizi{6-59}{经过}{}
\shizi{6-59}{路旁}{}
\shizi{6-59}{笑呵呵}{}
\shizi{6-59}{稀少}{}
\shizi{6-59}{骑自行车}{}
\shizi{6-59}{行走}{}
\shizi{6-59}{远方}{}
\shizi{6-59}{旁边}{}
\shizi{6-59}{大婶}{}
\shizi{6-59}{奇怪}{}
\shizi{6-59}{稀奇}{}
\shizi{6-59}{远近}{}
\shizi{6-59}{出远门}{}
\shizi{6-59}{旁人}{}
\shizi{6-59}{婶子}{}
\shizi{6-59}{骑马}{}
\shizi{6-59}{人行道}{}
\shizi{6-59}{走远}{}
\shizi{6-59}{一旁}{}
\shizi{6-59}{旁听}{}
\shizi{6-59}{乐呵呵}{}
\shizi{6-59}{稀饭}{}
}

% 宝宝读短文
\SecJuzi{
\shizi{6-59}{%两个苹果
小花狗骑着自行车回家,从车的后篮里掉下来两个大苹果。小羊看见了,喊道:“小花狗,苹果掉了!”喊了好几声,小花狗都没有听见。小羊只好拿着两个大苹果,看着小花狗骑远了。
小猪看见了,对小羊说:“小花狗已经走远了。就给我吃一个吧!”
小羊说:“不行。”
小猪说:“你一个人想吃两个。真坏!”
小白兔开着小汽车经过这里,小羊急忙叫道:“小白兔,小白兔,你停一停,帮我把大苹果还给小花狗。”
小白兔说:“你坐上来吧。我开车赶上小花狗,你自己还给他吧!”
小猪在一旁,听见他们说的花,脸儿红了。
小白兔开车赶上了小花狗,小花狗接过小羊的苹果,说:“谢谢小羊,谢谢你们大家。”}{}
}


\SecDanzi{
\shizi{6-60}{其}{}
\shizi{6-60}{实}{}
\shizi{6-60}{轻}{}
\shizi{6-60}{响}{}
\shizi{6-60}{迟}{}
\shizi{6-60}{已}{}
\shizi{6-60}{叠}{}
\shizi{6-60}{包}{}
\shizi{6-60}{碗}{}
\shizi{6-60}{写}{}
\shizi{6-60}{冒}{}
\shizi{6-60}{您}{}
\shizi{6-60}{了}{}
}

% 宝宝读词语
\SecCi{
\shizi{6-60}{打钟}{}
\shizi{6-60}{祝贺}{}
\shizi{6-60}{脑子}{}
\shizi{6-60}{左脑}{}
\shizi{6-60}{呆头}{}
\shizi{6-60}{打闹}{}
\shizi{6-60}{准备}{}
\shizi{6-60}{钟点}{}
\shizi{6-60}{贺信}{}
\shizi{6-60}{头脑}{}
\shizi{6-60}{小脑}{}
\shizi{6-60}{呆脑}{}
\shizi{6-60}{热闹}{}
\shizi{6-60}{准是}{}
\shizi{6-60}{拨动}{}
\shizi{6-60}{贺电}{}
\shizi{6-60}{大脑}{}
\shizi{6-60}{发呆}{}
\shizi{6-60}{起床}{}
\shizi{6-60}{闹钟}{}
\shizi{6-60}{准时}{}
\shizi{6-60}{贺礼}{}
\shizi{6-60}{右脑}{}
\shizi{6-60}{呆呆的}{}
\shizi{6-60}{小床}{}
\shizi{6-60}{看钟}{}
}


% 宝宝读短文
\SecJuzi{
\shizi{6-60}{%虎宝宝和小猫
虎妈妈生了个宝宝,接生的狐狸大婶乐呵呵地说:“看,虎宝宝真好玩,像只可爱的小猫。”
兔子正好经过这里,没有听清狐狸大婶的话,就跑去告诉松鼠:“稀奇!稀奇!虎妈妈生了一只小猫。”
松鼠想,是小猫一定会爬树。松鼠就跑去告诉小熊:“不稀奇!不稀奇!虎妈妈  生的小猫会爬树。”
小熊想 ,会爬树,也一定会捉老鼠。小
熊就跑去告诉小鹿:“稀奇!也不稀奇!虎妈妈生的小猫会爬到树上捉老鼠。”
于是,兔子、松鼠、小熊、小鹿一起去向虎妈妈祝贺。狐狸大婶抱出宝宝给大家一看,哟,虎头虎脑的,不是小猫呀!
这是怎么回事呢?小鹿怪小熊,小熊怪松鼠,松鼠怪兔子,兔子呆呆地站在那里,不知道怪谁。
你知道怪谁吗?}{}
}


% 宝宝读词语
\SecCi{
\shizi{6-复习8}{其中}{}
\shizi{6-复习8}{老实}{}
\shizi{6-复习8}{轻轻地}{}
\shizi{6-复习8}{轻松}{}
\shizi{6-复习8}{很响}{}
\shizi{6-复习8}{已经}{}
\shizi{6-复习8}{其实}{}
\shizi{6-复习8}{实话}{}
\shizi{6-复习8}{轻快}{}
\shizi{6-复习8}{响亮}{}
\shizi{6-复习8}{迟到}{}
\shizi{6-复习8}{以往}{}
\shizi{6-复习8}{其他}{}
\shizi{6-复习8}{果实}{}
\shizi{6-复习8}{轻声}{}
\shizi{6-复习8}{响声}{}
\shizi{6-复习8}{迟早}{}
\shizi{6-复习8}{叠起}{}
\shizi{6-复习8}{包子}{}
\shizi{6-复习8}{大碗}{}
\shizi{6-复习8}{饭碗}{}
\shizi{6-复习8}{写日记}{}
\shizi{6-复习8}{了不起}{}
\shizi{6-复习8}{叠被}{}
\shizi{6-复习8}{纸包}{}
\shizi{6-复习8}{水碗}{}
\shizi{6-复习8}{写信}{}
\shizi{6-复习8}{冒出来}{}
\shizi{6-复习8}{了不得}{}
\shizi{6-复习8}{书包}{}
\shizi{6-复习8}{小碗}{}
\shizi{6-复习8}{写字}{}
\shizi{6-复习8}{写书}{}
\shizi{6-复习8}{您好}{}
\shizi{6-复习8}{不得了}{}
\shizi{6-复习8}{了解}{}
\shizi{6-复习8}{了结}{}
}

% 宝宝读短文
\SecJuzi{
\shizi{6-复习8}{%闹钟响迟了
天黑了,兔宝宝上床睡觉了。兔妈妈吧闹钟拨到6点,准备早上起来给兔宝宝做早饭。
其实,兔宝宝一直没有睡着,他在等着妈妈睡觉。等到兔妈妈睡着了,兔宝宝就轻轻地起了床,把闹钟拨到了7点。
第二天早上,闹钟响了,兔妈妈坐起来一看,不得了!都7点了。兔宝宝会迟到的。
可是一看,兔宝宝床上的被子已经叠好了,书包也没有了。兔宝宝已经上学去了。
在桌子上放着一碗面条,一个荷包蛋,
还在冒热气呢。旁边有一张纸条,上面写着:“妈妈:能够帮您做点事,我很快乐!请吃早饭吧。我去上学了。”
你知道兔宝宝为什么要这样做吗?因为它觉得自己长大了,应该帮着妈妈做事了。}{}
}


\end{document}
